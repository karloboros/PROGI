\chapter{Zaključak i budući rad}
		
		 %\textit{U ovom poglavlju potrebno je napisati osvrt na vrijeme izrade projektnog zadatka, koji su tehnički izazovi prepoznati
		 %, jesu li riješeni ili kako bi mogli biti riješeni, koja su znanja stečena pri izradi projekta, koja bi znanja bila posebno 
		 %potrebna za brže i kvalitetnije ostvarenje projekta i koje bi bile perspektive za nastavak rada u projektnoj grupi.}
		 %\textit{Potrebno je točno popisati funkcionalnosti koje nisu implementirane u ostvarenoj aplikaciji.}
		 \noindent Zadatak naše grupe Program Tvog Kompjutera iz predmeta programsko inženjerstvo bio je napraviti web aplikaciju koja će 
		olakšati pronalazak plesnih klubova, tečajeva i plesnjaka u blizini korisnika.\\
		\noindent Izrada aplikacije trajala je otprilike 16 tjedana, a provedena je u dvije glavne faze koje su odvojene 
		unaprijed određenim rokovima projekata.\\
		\noindent Prva faza izrade aplikacije više je naglašavala izradu dokumentacije i dobru pripremu za izradu programske potpore. 
		Na samom početku okupljanja tima uslijedilo je upoznavanje i generalni dogovor i ideje oko mogućih tehnologija koje će biti korištene, načina komunikacije i sl.
		Nakon toga je uslijedio intenzivan rad na dokumentaciji, izradi modela baze podataka i napokon, implementacija određenog dijela obrazaca uporabe iz 
		dokumentacije. U početku je većini bilo nejasno zašto je tako veliki naglasak na dokumentaciji jer nam je, naravno, to bio dosadniji dio posla 
		koji nam je oduzimao dosta vremena jer se prije s njim nismo susreli. Kad smo krenuli s prvim implementacijskim detaljima shvatili smo zbog 
		čega se toliko zahtjevalo na dokumentaciji jer nam je ona poslužila kao odlična podloga za realizaciju programske potpore. Budući da smo 
		postavili dobre temelje sa obrascima uporabe, UML dijagramima i dijagramom baze podataka sve implementacijske zadatke je bilo lako i brzo 
		podijeliti. Pomoću dijagrama smo uočili ključne i kritične dijelove koje je potrebno realizirati prije, a koje kasnije kako nebi došlo do 
		problema.\\
		\noindent U drugoj fazi uslijedilo je intenzivno programiranje i implementacija programske potpore za ostatak obrazaca uporabe kako bi aplikacija 
		obuhvatila sve zadane funkcionalnosti. Generalno smo podijelili poslove po parovima kako bi zajedno radili, istraživali i međusobno se provjeravali. 
		U nekim slučajevima smo naišli na problem zato što smo posao u parovima podijelili tako da izrada dijela programa jednog para ovisi o izradi dijela drugog para što 
		nije bilo najidelanije rješenje zato što bi se u nekim trenucima morali čekati i parovi bi jako ovisili jedan o drugome. Jasno je da rad u timu 
		sa sobom implicira takve probleme, no kad smo to primjetili, pokušali smo dogovoriti da jedna osoba radi zadatak u cjelini što se pokazalo kao puno 
		brža opcija i jednostavnija za testiranje rješenja. Osim toga, nakon realizacije je bilo potrebno izraditi ostatak dokumentacije kako bi aplikacija 
		imala dobre temelje za održavanje.\\
		\noindent Ulaskom u ovaj projekt imali smo različita iskustva i znanja s novim tehnologijama što nas je primoralo na 
		dodatno istraživanje. Jednostavni zadaci su u početku trajali puno duže i bili su vrlo iscrpni tako dugo dok se nismo uhodali i samim time postali brži. 
		Iskusniji članovi koji su se već susreli s potencijalnim problemima drugih uvijek su bili spremni ukazati na moguća rješenja i pomoći.\\
		\noindent Svaki zadatak i merge request bio je popraćen detaljnim pregledom drugog člana tima koji je ukazao na moguće greške i bolja rješenja što nam je 
		jako pomoglo da cijeli projekt bude konzistentan te da se potencijalne greške uoče na vrijeme.\\
		\noindent Komunikacija grupe bila je na Slack-u gdje smo obavještavali ostale članove o zadacima na kojima radimo, stanju zadataka, problemima i sl.\\
		\noindent Obzirom na ostale fakultetske obaveze vrlo smo zadovoljni ostvarenim ciljem. Raznolikost u iskustvima nam je puno pomogla u ostvarivanju ciljeva i 
		možemo reći da se nakon ovog projekta osjećamo zrelije u izradi web aplikacija. Svatko od nas se susreo s nečim novim i našao način za rješenje problema. 
		Osjetili smo što znači raditi u timu i kakve organizacijske odgovornosti nosi projekt na kojem radiš od definiranja zahtjeva do produkcije.\\
		\eject 