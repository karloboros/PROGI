\chapter{Dnevnik promjena dokumentacije}
		
		\textbf{\textit{Kontinuirano osvježavanje}}\\
				
		
		\begin{longtblr}[
				label=none
			]{
				width = \textwidth, 
				colspec={|X[2]|X[13]|X[3]|X[3]|}, 
				rowhead = 1
			}
			\hline
			\textbf{Rev.}	& \textbf{Opis promjene/dodatka} & \textbf{Autori} & \textbf{Datum}\\[3pt] \hline
			0.1 & Napravljen predložak.	& Lara Đaković & 30.10.2022. 		\\[3pt] \hline
			0.2 & Dodana 2 \textit{Use Case} dijagrama - admin i vlasnik kluba te opisi za dio obrazaca uporabe& Lara Đaković & 1.11.2022. \\[3pt] \hline 
			0.3 & Dodan opis projektnog zadatka & Nina Đurić & 1.11.2022. \\[3pt] \hline
			0.4 & Dodani funkcionalni zahtjevi & Mateja Golec & 2.11.2022. \\[3pt] \hline 
			0.5 & Dodan \textit{Use Case} dijagram - neregistrirani korisnik, klijent i trener te opisi za dio obrazaca uporabe& Ana Vrabec & 3.11.2022.\\[3pt] \hline 
			0.6 & Dodan UC za neregistriranog korisnika i admina & Lucija Domić & 3.11.2022.\\[3pt] \hline 			
			0.7 & Ispravak obrazaca uporabe i dijagrama obrazaca uporabe za neregistriranog korisnika, klijenta i trenera & Ana Vrabec & 4.11.2022.\\[3pt] \hline 	
			0.8 & Dodan sekvencijski dijagram Prijava trenera i funkcionalnosti & Mateja Golec & 7.11.2022. \\[3pt] \hline
			0.9 & Dodani opisi obrazaca uporabe za klub & Karlo Boroš & 5.11.2022. \\[3pt] \hline
			0.10 & Dodan opis baze podataka i opis tablica Tecaj, Ples, Lokacija, PlesnjakPles i Trening & Lara Đaković & 7.11.2022. \\[3pt] \hline 	
			0.11. & Dodani sekvencijski dijagrami Upravljanje tečajem i Upravljanje plesovima & Nina Đurić & 9.11.2022. \\[3pt] \hline
			0.12 & Dodani ostali zahtjevi & Ana Vrabec & 9.11.2022. \\[3pt] \hline
			0.13 & Dodani opisi tablica za Klub, Korisnik, TrenerPrijava, Plesnjak, KorisnikTecaj & Karlo Boroš & 9.11.2022. \\[3pt] \hline	
			\textbf{1.0} & Verzija samo s bitnim dijelovima za 1. ciklus & * & 18.11.2022. \\[3pt] \hline 
			
		\end{longtblr}
	
	
		\textit{Moraju postojati glavne revizije dokumenata 1.0 i 2.0 na kraju prvog i drugog ciklusa. Između tih revizija mogu postojati manje revizije već prema tome kako se dokument bude nadopunjavao. Očekuje se da nakon svake značajnije promjene (dodatka, izmjene, uklanjanja dijelova teksta i popratnih grafičkih sadržaja) dokumenta se to zabilježi kao revizija. Npr., revizije unutar prvog ciklusa će imati oznake 0.1, 0.2, …, 0.9, 0.10, 0.11.. sve do konačne revizije prvog ciklusa 1.0. U drugom ciklusu se nastavlja s revizijama 1.1, 1.2, itd.}
